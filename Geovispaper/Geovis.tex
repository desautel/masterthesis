% This is "sig-alternate.tex" V2.1 April 2013
% This file should be compiled with V2.5 of "sig-alternate.cls" May 2012
%
% This example file demonstrates the use of the 'sig-alternate.cls'
% V2.5 LaTeX2e document class file. It is for those submitting
% articles to ACM Conference Proceedings WHO DO NOT WISH TO
% STRICTLY ADHERE TO THE SIGS (PUBS-BOARD-ENDORSED) STYLE.
% The 'sig-alternate.cls' file will produce a similar-looking,
% albeit, 'tighter' paper resulting in, invariably, fewer pages.
%
% ----------------------------------------------------------------------------------------------------------------
% This .tex file (and associated .cls V2.5) produces:
%       1) The Permission Statement
%       2) The Conference (location) Info information
%       3) The Copyright Line with ACM data
%       4) NO page numbers
%
% as against the acm_proc_article-sp.cls file which
% DOES NOT produce 1) thru' 3) above.
%
% Using 'sig-alternate.cls' you have control, however, from within
% the source .tex file, over both the CopyrightYear
% (defaulted to 200X) and the ACM Copyright Data
% (defaulted to X-XXXXX-XX-X/XX/XX).
% e.g.
% \CopyrightYear{2007} will cause 2007 to appear in the copyright line.
% \crdata{0-12345-67-8/90/12} will cause 0-12345-67-8/90/12 to appear in the copyright line.
%
% ---------------------------------------------------------------------------------------------------------------
% This .tex source is an example which *does* use
% the .bib file (from which the .bbl file % is produced).
% REMEMBER HOWEVER: After having produced the .bbl file,
% and prior to final submission, you *NEED* to 'insert'
% your .bbl file into your source .tex file so as to provide
% ONE 'self-contained' source file.
%
% ================= IF YOU HAVE QUESTIONS =======================
% Questions regarding the SIGS styles, SIGS policies and
% procedures, Conferences etc. should be sent to
% Adrienne Griscti (griscti@acm.org)
%
% Technical questions _only_ to
% Gerald Murray (murray@hq.acm.org)
% ===============================================================
%
% For tracking purposes - this is V2.0 - May 2012

\documentclass{acm_proc_article-sp}


\usepackage{epstopdf}

\begin{document}

% Copyright
%\setcopyright{acmcopyright}
%\setcopyright{acmlicensed}
%\setcopyright{rightsretained}
%\setcopyright{usgov}
%\setcopyright{usgovmixed}
%\setcopyright{cagov}
%\setcopyright{cagovmixed}




\title{Real-Time Dynamic Global Illumination}
%\subtitle{[Extended Abstract]}
%
% You need the command \numberofauthors to handle the 'placement
% and alignment' of the authors beneath the title.
%
% For aesthetic reasons, we recommend 'three authors at a time'
% i.e. three 'name/affiliation blocks' be placed beneath the title.
%
% NOTE: You are NOT restricted in how many 'rows' of
% "name/affiliations" may appear. We just ask that you restrict
% the number of 'columns' to three.
%
% Because of the available 'opening page real-estate'
% we ask you to refrain from putting more than six authors
% (two rows with three columns) beneath the article title.
% More than six makes the first-page appear very cluttered indeed.
%
% Use the \alignauthor commands to handle the names
% and affiliations for an 'aesthetic maximum' of six authors.
% Add names, affiliations, addresses for
% the seventh etc. author(s) as the argument for the
% \additionalauthors command.
% These 'additional authors' will be output/set for you
% without further effort on your part as the last section in
% the body of your article BEFORE References or any Appendices.

\numberofauthors{1} %  in this sample file, there are a *total*
% of EIGHT authors. SIX appear on the 'first-page' (for formatting
% reasons) and the remaining two appear in the \additionalauthors section.
%
\author{
% You can go ahead and credit any number of authors here,
% e.g. one 'row of three' or two rows (consisting of one row of three
% and a second row of one, two or three).
%
% The command \alignauthor (no curly braces needed) should
% precede each author name, affiliation/snail-mail address and
% e-mail address. Additionally, tag each line of
% affiliation/address with \affaddr, and tag the
% e-mail address with \email.
%
% 1st. author
\alignauthor
Johannes Linke\\
       \affaddr{Hasso Plattner Institute}\\
       %\affaddr{1932 Wallamaloo Lane}\\
       %\affaddr{Wallamaloo, New Zealand}\\
       \email{johannes.linke@student.hpi.de}
}

\maketitle
\begin{abstract}
TODO
\end{abstract}



\section{Introduction}
For decades, computer graphics researchers have strived to achieve the faithful reproduction of reality in synthetic renderings. While photorealism has been achieved in offline rendering contexts, many optical effects occurring in physical environments are still not in use in interactive and real-time applications or implemented with severe limitations. Several lighting effects collectively referred to as ``global illumination'', such as caustics, subsurface scattering and diffuse and specular indirect light, fall into this category.

In this paper, we develop a system that simulates diffuse indirect light in arbitrary and dynamic scenes using many-light techniques. We will trade in quality and instead focus on performance while maintaining scalability with the goal to achieve real-time performance on commodity hardware.




\section{Related Work}

Many-light methods for simulating global illumination effects are a well-researched topic. They are based on Instant Radiosity \cite{Keller:1997:InstantRadiosity}, in which photon mapping is used to create small lights, called Virtual Point Lights (VPLs) at every intersection of photons with the scene geometry.


Most of the work on many-light methods can be categorized into the areas VPL placement, solving occlusion, gathering light into the framebuffer and mitigating singularities, an artifact that occurs near positions of VPLs.


The foundation of many algorithms for VPL placment are Reflective Shadow Maps (RSMs) \cite{Dachsbacher:2005:RSM}, which are shadow maps with additional surface information. Using the additional data, VPLs are created per texel of the RSM. View-adaptive placement of VPLs greatly enhances performance and quality \cite{ritschel2011ismsViewAdaptive}. \cite{prutkin2012reflective} cluster RSM samples to virtual area lights, and \cite{hedman2016sequential} focuses on temporal coherence when placing VPLs to avoid flickering.

A common approach for solving occlusion are Imperfect Sha\-dow Maps (ISMs) \cite{ritschel2008ism}, which use a point-based approximation of the scene to efficiently render a small, approximate shadow map for hundreds of VPLs simultaneously. ISMs have been used in production for rendering many spotlights with shadows \cite{evans2015dreams}, and several enhancements have been proposed \cite{ritschel2011ismsViewAdaptive, hollander2011manylods, barak2013temporally}. Using slim voxels \cite{sugihara2014layered, sun2015manylightsSVO, chen2016quantizing} for occlusion is another approach that needs only a single bit per voxel for visibility testing, thereby avoiding the usually high memory requirements of voxelization techniques.

\cite{dachsbacher2006splatting, Nichols:2009:splatting} are using splatting techniques in favor of the more common gathering approach. \cite{sloan2007image, laine2007incremental} provide details on how to speed up the straight-forward approach of iterating over sets of VPLs per fragment.

In na\"ive implementations of many-light methods, bright spots will appear near the VPL's positions due to the light's attenuation term approaching infinity. A common approach is to clamp the term. This introduces bias, which can be compensated e.g. in screen space \cite{novak2011screen}. Those singularities can also be avoided through more advanced light representations \cite{tokuyoshi2015vsgl}. \cite{olsson2012clustered}



%
% The following two commands are all you need in the
% initial runs of your .tex file to
% produce the bibliography for the citations in your paper.
\bibliographystyle{abbrv}
\bibliography{sigproc}  % sigproc.bib is the name of the Bibliography in this case
% You must have a proper ".bib" file
%  and remember to run:
% latex bibtex latex latex
% to resolve all references
%
% ACM needs 'a single self-contained file'!
%
%APPENDICES are optional
%\balancecolumns


%\balancecolumns % GM June 2007
% That's all folks!
\end{document}
