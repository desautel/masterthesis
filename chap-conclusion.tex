%!TEX root = foo-thesis.tex


\chapter{Conclusion and Future Work}
\label{chap:conclusion}

In this thesis we have presented a rendering system that simulates global illumination at real-time framerates. It takes only a few milliseconds for computing indirect lighting, making it suitable for use in real-time applications on today's hardware.

In addition to imperfect shadow maps rendered with splatting, a second renderer based on compute shaders has been implemented. With the flexibility provided by compute shaders, it is able to render a high amount of points more efficiently compared to the splat renderer, which relies on geometry shaders. Extended with a high-quality postprocessing, it renders shadow maps of noticeable higher quality than the splat renderer, but the resulting quality is limited by the input data, a randomly selected and sparse set of points representing the scene. Several  failure cases have been identified where the presented implementation generates artifacts clearly visible even to the untrained eye.

Interleaved shading has traditionally been performed by the means of several shader passes, which split a buffer into several smaller ones, perform processing on them, and merge these buffers again. This thesis has presented an implementation that is integrated into the final gathering shader, making splitting and merging buffers obsolete while still processing the workload cache-coherently. Its near-perfect efficiency, ease of implementation, and the fact that it requires no additional memory make this implementation the obvious choice when compute shaders are available on the given platform.

Two light culling techniques well-known in the video games industry, tiled and clustered deferred shading, have been applied to many-light global illumination. While tiled deferred shading's simplicity allowed for a compact implementation integrated into the final gathering shader, clustered deferred shading promised higher culling efficiency at the expense of higher overhead. The results have shown this overhead to be not worth the gains, making clustered deferred shading not the perfect candidate for light culling in a global illumination context using many-light methods. There are extensions, however, that could improve its culling efficiency to the point where it causes no slowdown in the worst case. Tiled deferred shading, on the other hand, is a clear performance win and again, rather easy to implement, but limited in its maximum culling efficiency.


Looking forward, going beyond random point sampling for creating imperfect shadow maps is desirable, since the randomness is limiting their quality. The implementation of more advanced surface reconstruction techniques, starting with the full extent of the algorithm presented by \citet{Marroquim:2007:reconstruction}, would likely further enhance the ISM's quality.

Given how common the usage of interleaved sampling is, it is surprising that little work has been done to maximize the technique's impact. For instance, thoroughly investigating the correlation between performance and quality when changing the block size would give implementers a guideline on which block sizes to choose in which scenarios.

There is still potential in the clustered deferred shading technique, for instance by using surface normals for more efficient culling. Mipmapping the ISMs and using them for culling, performing an early shadow lookup for entire clusters of fragments, would be interesting as well.

However, the final gathering phase is quite fast at least on high-end hardware, and, more importantly,  not detrimental to quality. While rendering the ISMs is fast as well, their quality is lacking and limiting the potential use of the presented technique. Improving their precision by, for example, improving the point sampling or surface reconstruction or investigating triangle-based approaches, is in our view the path with the largest potential benefit.
